%
% File - ch1.tex
%
%
% Description:
% This file should contain the first real chapter or section of your
% thesis.
%
%
The geometry of two dominoes, separated by a distance $d$, is
shown in Figure \ref{fig:geometry}. The angular displacement of a
single domino is denoted $\theta$, measured from the upright
position, and it is governed by the familiar pendulum equation

\begin{equation} \label{eq:eq1}
\theta''(t) - k^2 \sin \theta(t)=0,
\end{equation}

where $k^2=\frac{3g}{2h}$, $g$ is the acceleration due to gravity,
and $h$ is the height of the domino. The factor of $\frac 32$ in
$k$ accounts for the moment of inertia of a thin rod, and the
minus sign in this equation is due to the fact that $\theta$ is
measured from the upright position. The critical angle at which
one domino strikes its neighbor is given by $\theta^\ast =
\sin^{-1} (\frac{d}{h})$. The striking point is a distance
$y=\sqrt{h^2-d^2}$ from the ground.\cite{golub}

\begin{figure}[ht]
\begin{center}
%You can include postscript graphics files here and size them.
%\includegraphics[height=3.0in,width=5.0in]{dom2.eps}
\vspace{0.5in}

{\sc FIGURE GOES HERE}

\vspace{0.5in}

\end{center}
\caption{\label{fig:geometry}{\em Two successive dominoes in the
chain are separated by a distance $d$. The critical angle at which
the first domino strikes its neighbor is given by $\theta^\ast =
\sin^{-1} (\frac{d}{h})$.}}
\end{figure}

Multiplying (\ref{eq:eq1}) by $\theta'$ allows us to write
\begin{equation} \label{eq:eq2}
\frac 12 \frac{d}{dt}(\theta'(t))^2 - k^2 \sin
\theta(t)\theta'(t)=0.
\end{equation}
Integrating (\ref{eq:eq2}), letting $\theta'(0)=\omega_1$, and
noting that $\theta(0)=0$, leads to
\[
(\theta'(t))^2-\omega_1^2 = 2k^2(1-\cos \theta(t)) .
\]
After some rearrangement, we find that the angular velocity of the
domino is given by\cite{brandtmoc}
\begin{equation} \label{eq:eq3}
\theta'(\theta)=\sqrt{\frac{3g(1-\cos \theta) + h \omega_1^2}{h}}.
\end{equation}
The angular velocity equation can also be derived from energy
conservation arguments.

Two important calculations result from the angular velocity
equation. First, we assume that only the horizontal component of
the angular velocity at impact is imparted to the next domino.
Using $\cos \theta^\ast = \frac yh$, this means that the initial
velocity of the next domino as it begins its fall is
\[
\omega_2 = \theta'(\theta^\ast) \cos \theta^\ast =
\frac{y}{h^2}\sqrt{3g(h-y)+h^2 \omega_1^2}.
\]


Local mode analysis can be extended easily to two or more
dimensions. In two dimensions, the Fourier modes have the form
\begin{equation}\label{eq:modeform2d}
e_{jk}^{(m)} = A(m)e^{i (\theta_1 + \theta_2)},
\end{equation}
where $-\pi < \theta_1,\theta_2 \le \pi$ are the wavenumbers in
the $x$- and $y$-directions, respectively. Substituting this
representation into the error updating step generally leads to an
expression for the change in the amplitudes of the form
\[
A(m+1) = G(\theta_1,\theta_2)A(m).
\]
The amplification factor now depends of two wavenumbers. The
smoothing factor is the maximum magnitude of the amplification
factor over the oscillatory modes. As we see in Fig.
\ref{fig:osc2d}, the oscillatory modes correspond to $\pi/2 \le
|\theta_i| \le \pi$ for either $i=1$ or $i=2$; that is,
\[
\mu = \max_{\pi/2 \le |\theta_i| \le \pi}|G(\theta_1,\theta_2)|.
\]

%%%%%%%%%%%%%%%%%%%%% Figure created within LaTeX.

\begin{figure}
\begin{center}
\begin{picture}(250,250)(-25,-25)
\put(0,0){\line(1,0){200}} \put(200,0){\line(0,1){200}}
\put(0,0){\line(0,1){200}} \put(0,200){\line(1,0){200}}
% Don't know why this causes an overfill
\put(50,50){\dashbox{5}(100,100)}

\put(100,-10){\line(0,1){220}} \put(-10,100){\line(1,0){220}}
\put(105,105){$(0,0)$} \put(-25,-15){$(-\pi,-\pi)$}
\put(-15,210){$(-\pi,\pi)$} \put(195,210){$(\pi,\pi)$}
\put(195,-15){$(\pi,-\pi)$} \put(215,98){$\theta_1$}
\put(100,212){$\theta_2$}
\end{picture}
\end{center}
\caption{\label{fig:osc2d}{\em The oscillatory modes in two
dimensions correspond to the wave numbers $\pi/2 \le |\theta_i| <
\pi$ for either $i=1$ or $i=2$; this is the region outside of the
dashed box. }}.
\end{figure}

Now let's change the subject.

\begin{thm}
Let $f$ be continuously differentiable on $[a,b]$. Then there
exists $\xi \in (a,b)$ such that $f'(\xi)(b-a)= f(b)-f(a)$.
\end{thm}

\begin{proof} Intuitively clear.
\end{proof}

\begin{cor}
Let $f$ be continuously differentiable on $[a,b]$ with
$f(a)=f(b)=0$. Then there exists $\xi \in (a,b)$ such that
$f'(\xi)=0$.
\end{cor}

\begin{proof} Even more obvious.
\end{proof}
